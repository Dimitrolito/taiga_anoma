\documentclass{beamer}
\usepackage[utf8]{inputenc}
\usepackage{amsmath}
\usepackage{amssymb}
\usepackage{mathrsfs}
\usepackage{xcolor}

\renewcommand<>\cellcolor[1]{\only#2{\beameroriginal\cellcolor{#1}}}

\title{Addresses in PBC}

\begin{document}

\frame{\titlepage}

\begin{frame}\centering
    \begin{tabular}{|l|l|l|}
        \hline
        & \textbf{Old note} & \textbf{New note}\\
        \hline
        Owner address & \texttt{0x91a2} & \texttt{0x5e2b} \\
        \hline
        Token address & \cellcolor{red}<2>{\texttt{0xca6f}} & \texttt{0x2cd0} \\
        \hline
        Value & $12\text{XAN}$ & $1\text{BTC}$ \\
        \hline
        Nullifier & $(123,321)$ & -- \\
        \hline
    \end{tabular}

    \begin{itemize}
        \item \texttt{TokenVP} for XAN (\texttt{0xca6f}): "Accept sending maximum 10 XAN". This is a circuit.
        \item Verifying an XAN VP proof is done with \texttt{XAN.DescTokenVP}.
        \item The sender produces a proof for the XAN VP.
        \item $\text{XAN.Adress} = \text{Com}_q(XAN.DescTokenVP, XAN.rcm_addr)$. 
        \item Different addresses for different tokens: \texttt{rcmAddr}.
    \end{itemize}
\end{frame}

\end{document}
